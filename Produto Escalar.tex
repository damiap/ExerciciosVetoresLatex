\documentclass[a4paper,10pt]{article}
\usepackage[utf8]{inputenc}
\usepackage[brazilian]{babel}
\usepackage{amsmath}
\usepackage{enumerate}


%opening
\title{Primeiros Exercícios}
\author{}

\begin{document}

\maketitle

\section{Operações com Vetores}

\subsection{Soma e multiplicação por escalar}

\subsubsection{}Seja $\vec{A}=(1,3,6)$, $\vec{B}=(4,-3,3)$ e $\vec{C}=(2,1,5)$ três vetores. Determine as componentes de:

\begin{enumerate}
 \item $A + B$ = 
 (5, 0, 9) 
 
 \item $A - B$ = 
 (-3, 6, 3)
 
 \item $A + B - C$ =
  $(5, 0, 9) - (2, 1, 5)$ = (3, -1, 4)
 
 \item $7A - 2B - 3C$ = 
 (-7, 24, 21)
 
 \item $2A + B - 3C$ = 
 $(6, 3, 15) - (6, 3, 15)$ = (0, 0, 0)

\end{enumerate}

\subsubsection{}Seja $\vec{A}=(1,1,1)$, $\vec{B}=(0,1,1)$ e $\vec{C}=(1,1,0)$ três vetores, seja 
$\vec{D}=x\vec{A} + y\vec{B} + z\vec{C}$ onde $x$, $y$ e $z$ são escalares


\begin{enumerate}
 \item Determine as componentes de D(não tem todos os números o resultado fica em função de $x$, $y$ e $z$)   
 \newline
 $D = (x + z)\hat{i} + (x + y + z)\hat{j} + (x + y)\hat{k}$
 
 
  
 \item Se $\vec{D}=(0,0,0)$, mostre que $x=y=z=0$ 
 \newline
 ver caderno
 
 \item Encontre $x$, $y$ e $z$ tais que $\vec{D}=(1,2,3)$ 
 \newline
 x = 1
 y = 2
 z = 0
 
\end{enumerate}

\subsection{Produto Escalar}

\subsubsection{}Calcule $\vec{a} \cdot \vec{b}$ para:
%(\textit{Stewart 12.3.1})

\begin{enumerate}[a)]
 \item $\vec{a}=(2,4)$ e $\vec{b}=(3,1)$ = 
 10
 \item $\vec{a}=(-1,7,4)$ e $\vec{b}=(6,2,\frac{1}{2})$ = 
 10
 \item $\vec{a}= \hat{i} + 2\hat{j} - 3\hat{k}$ e $\vec{b}= 2\hat{j} - \hat{k}$ =
 7
\end{enumerate}


\subsubsection{}Dados os vetores $\vec{A}=(2,4,-7)$ $\vec{B}=(2,6,3)$ e $\vec{C}=(3,4,-5)$. 
Em cada uma das expressões abaixo existe apenas uma única maneira de colocar parenteses e obter uma
expressão que faça sentido. Coloque os parenteses e faça as operações: 
%(\textit{apostol 12.8.2})


\begin{enumerate}[a)]
 \item $ \vec{A} \cdot \vec{B}       \vec{C}$ = 
 $\vec{A}.(\vec{B}.\vec{C})$ = $-10$
 
 \item $ \vec{A} \cdot \vec{B} +     \vec{C}$ = 
 = $86$
 \item $ \vec{A} +     \vec{B} \cdot \vec{C}$ =
 $(\vec{A}+\vec{B})\vec{C}$ = 72
 \item $ \vec{A}       \vec{B} \cdot \vec{C}$ =
 $(\vec{A}\vec{B})\vec{C}$ = 62
 \item $ \vec{A} /     \vec{B} \cdot \vec{C}$ =
 $(\vec{A} / \vec{B})\vec{C}$  = $52 / 3$
\end{enumerate}

\subsubsection{}Sejam $\vec{a} = (a_1,a_2,a_3)$ $\vec{b} = (b_1,b_2,b_3)$ e $\vec{c} = (c_1,c_2,c_3)$ 

\begin{enumerate}[a)]
 \item $ \vec{a} \cdot (\vec{b}  +   \vec{c})$ = 
 $a1b1 + a1c1 + a2b2 + a2c2 + a3b3 + a3c3$

 \item  Com base no resultado obtido no item \textit{a}, podemos realizar a distributiva do produto escalar, ou seja
 $ \vec{a} \cdot (\vec{b}  +   \vec{c}) = \vec{a} \cdot \vec{b} + a \cdot c $ ? (se necessário manipule um pouco mais a expressão do item a) = 
 Sim, podemos realizar a distributiva do produto escalar, pois a resposta será a mesma do item anterior $a1b1 + a1c1 + a2b2 + a2c2 + a3b3 + a3c3$
\end{enumerate}

 \subsubsection{}Se $\vec{A} = (2,1,-1)$ e $\vec{B} = (1,-1,2)$ ache um vetor não nulo $\vec{C}$  tal que 
 $\vec{A} \cdot \vec{C} = \vec{B} \cdot \vec{C} = 0$  = 
 $\vec{C} = (c1, 5.c1, -3.c1)$
 %apostol 12.8.5
 
 
 





\end{document}
